\documentclass{article}
\title{Shunbo's 2AA4 learning log}
\author{Shunbo Cui}
\date{}
\begin{document}
\maketitle

\begin{flushleft}
Jan 13, 2019
\end{flushleft}
\par
In the first week of the winter semester I learned some basic knowledge about more useful software. In the tutorial there was a detailed and clear demonstration about the installation of LaTeX. This is a powerful document preparation system for automatically formatting the articles, which can make them organized and stable. In my future work I believe it will be widely used in my reports and resume.  \par
We also went through some commands of git in the tutorial. It is a version control tool like SVN I used in the last semester. By committing and rushing I can update and save the changes on files to the gitlab.\par
In the assignment I worked on how to read data from files with python. The main idea is use the function readlines and save the data in each line in a dictionary. I also need to sort the dictionaries according to the value of gpa.

\begin{flushleft}
Jan 20, 2019
\end{flushleft}
\par
In this week I mainly added more details to the assignment. For test cases I found it actually hard to cover all the possible cases. It was easier to test the average calculation and the sorting function. I needed to test cases for both gender, and the empty student set. However for the allocating function there are more possible cases. The number of students with free choice can exceed the limit of department capacity, and so as the normal students. When the capacity is full I have to consider their second choice. So the function needs to work in different conditions.\par

\begin{flushleft}
Jan 27, 2019
\end{flushleft}
\par
I got the partner file and tested it with my test driver in assignment 1 this week. As I said in my report, many of my tests did not pass for the partner;s functions. The main reason is the style of the code. The code is general and correct, but some difference in detail may cause the fail in test. We also covered some relations between discrete math and software language, such as the use of logic symbols.\par

\begin{flushleft}
Feb 3, 2019
\end{flushleft}
\par
In this week we went through the concept of MIS. It specificates how the module should work, which enables working in parallel. And it enables the code easy and clear to read or modify. It includes local functions, types and constants.

\begin{flushleft}
Feb 10, 2019
\end{flushleft}
\par
In this week I worked on the assignment which implements ADT. It creates the module communicating with the user. Every module is defined in seperate classes so they are conveinent to modify. I used flake 8 in my codes to help ensure the codes meets the style requirement.

\begin{flushleft}
Feb 17, 2019
\end{flushleft}
\par
We covered Object Oriented Design this week. A class of Point ADT is given as example. A class exports operations (procedures) to manipulate instance objects. It can have multiple instances of the class (class can be thought of as roughly corresponding to the notion of a type). A class can specialize another class.

\begin{flushleft}
Feb 24, 2019
\end{flushleft}
\par
Exceptions always exists in design process. We need to decide how exception signalling will be done. A special return value, a special status parameter, a global variable and invoking an exception procedure. It uses built-in language constructs. They are caused by errors mady by programmers. It is particular useful in testing the code.


\begin{flushleft}
Mar 3, 2019
\end{flushleft}
\par
This week we covered module decomposition. We need to decompose the system into modules, assign responsibilities to those modules and ensure that they fit together to achieve our global goals. Decomposing a large software system into modules enables us to focus on individual modules. We need to know the Anticipate definition of all family members and identify what is common to all family members, delay decisions that differentiate among different members.

\begin{flushleft}
Mar 10, 2019
\end{flushleft}
\par
We covered maze tracing robot MG example specification this week. The main functions are initializing, starting and ending. Specification is mainly used as a statement of user requirements; the interface between the machine and the controlled environment; the requirements for the implementation and reference point during product maintenance. It needs to be clear, unambiguous, understandable, and also complete and abstract.


\begin{flushleft}
Mar 17, 2019
\end{flushleft}
\par
We learned the important steps of translating English problem to math problem. They both have rule of grammar and semantics. The difference is that in English, ambiguity is desired. In mathmetics ambiguous statements are almost not existing. The first step is to understand the meaning of the original. Then get the needed information and close the gap between different languages.


\begin{flushleft}
Mar 24, 2019
\end{flushleft}
\par
I mainly learned how to implement C++ codes according to the given specification. The use of class is a lot different from Python. I need to define every function and variable first in the header file. The use of templete Stack is also new technique for me. I define a templete class of Stack first, and then I can use it in other classes.
\end{document}

\begin{flushleft}
Mar 31, 2019
\end{flushleft}
\par
This week we covered the importance of testing in software designing. It is used to show the presense of bugs. It should be repeatable and accurate, which means we need to get the same result each time we run the same experienment. Also white box and black box testing are covered. The main difference between them is either how processing occurs is concerned.
\end{document}
