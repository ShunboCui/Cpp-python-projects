\documentclass{article}
\title{Shunbo's 2AA4 learning log}
\author{Shunbo Cui}
\date{}
\begin{document}
\maketitle

\begin{flushleft}
Jan 13, 2019
\end{flushleft}
\par
In the first week of the winter semester I learned some basic knowledge about more useful software. In the tutorial there was a detailed and clear demonstration about the installation of LaTeX. This is a powerful document preparation system for automatically formatting the articles, which can make them organized and stable. In my future work I believe it will be widely used in my reports and resume.  \par
We also went through some commands of git in the tutorial. It is a version control tool like SVN I used in the last semester. By committing and rushing I can update and save the changes on files to the gitlab.\par
In the assignment I worked on how to read data from files with python. The main idea is use the function readlines and save the data in each line in a dictionary. I also need to sort the dictionaries according to the value of gpa.

\begin{flushleft}
Jan 20, 2019
\end{flushleft}
\par
In this week I mainly added more details to the assignment. For test cases I found it actually hard to cover all the possible cases. It was easier to test the average calculation and the sorting function. I needed to test cases for both gender, and the empty student set. However for the allocating function there are more possible cases. The number of students with free choice can exceed the limit of department capacity, and so as the normal students. When the capacity is full I have to consider their second choice. So the function needs to work in different conditions.\par

\begin{flushleft}
Jan 27, 2019
\end{flushleft}
\par
I got the partner file and tested it with my test driver in assignment 1 this week. As I said in my report, many of my tests did not pass for the partner;s functions. The main reason is the style of the code. The code is general and correct, but some difference in detail may cause the fail in test. We also covered some relations between discrete math and software language, such as the use of logic symbols.\par

\begin{flushleft}
Jan 20, 2019
\end{flushleft}
\par
In this week we went through the concept of MIS. It specificates how the module should work, which enables working in parallel. And it enables the code easy and clear to read or modify. It includes local functions, types and constants.

\begin{flushleft}
Jan 20, 2019
\end{flushleft}
\par
In this week I worked on the assignment which implements ADT. It creates the module communicating with the user. Every module is defined in seperate classes so they are conveinent to modify. I used flake 8 in my codes to help ensure the codes meets the style requirement. 
\end{document}
