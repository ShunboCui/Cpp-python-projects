\documentclass[12pt]{article}

\usepackage{graphicx}
\usepackage{paralist}
\usepackage{amsfonts}
\usepackage{amsmath}
\usepackage{hhline}
\usepackage{booktabs}
\usepackage{multirow}
\usepackage{multicol}

\oddsidemargin 0mm
\evensidemargin 0mm
\textwidth 160mm
\textheight 200mm
\renewcommand\baselinestretch{1.0}

\pagestyle {plain}
\pagenumbering{arabic}

\newcounter{stepnum}

%% Comments

\usepackage{color}

\newif\ifcomments\commentstrue

\ifcomments
\newcommand{\authornote}[3]{\textcolor{#1}{[#3 ---#2]}}
\newcommand{\todo}[1]{\textcolor{red}{[TODO: #1]}}
\else
\newcommand{\authornote}[3]{}
\newcommand{\todo}[1]{}
\fi

\newcommand{\wss}[1]{\authornote{blue}{SS}{#1}}

\title{Assignment 4 Specification}
\author{SFWR ENG 2AA4}

\begin {document}

\maketitle

\noindent This Module Interface Specification (MIS) document contains modules, types and methods for implementing the game state of
The Game of Life. In each state of the game, the cell on the game board can be populated or empty and according to the number of neighbours, the population of cells can be changed in the next state.
\begin{center}

\end{center}

\newpage

\section* {File Module}

\subsection*{Module}

File

\subsection* {Uses}

N/A

\subsection* {Syntax}

\subsubsection* {Exported Constants}

None


\subsubsection* {Exported Access Programs}

\begin{tabular}{| l | l | l | l |}
\hline
\textbf{Routine name} & \textbf{In} & \textbf{Out} & \textbf{Exceptions}\\
\hline
File & string & File & ~\\
\hline
read & ~ & seq of (seq of $\mathbb{Z}$) & ~\\
\hline
write & seq of (seq of $\mathbb{Z}$) & ~ & ~\\

\hline
\end{tabular}

\subsection* {Semantics}

\subsubsection* {State Variables}

filename

\subsubsection* {State Invariant}

None
\subsubsection* {Access Routine Semantics}

File($filename$):
\begin{itemize}
\item transition: $filename := filename$
\item exception: None
\end{itemize}

\noindent read():
\begin{itemize}
\item output: $out := total$
\item exception: None
\end{itemize}

\noindent write(total):
\begin{itemize}
\item output: None
\item exception: None
\end{itemize}

\subsubsection* {Assumptions \& Design Decisions}
\begin{itemize}
\item
This module is used to read an input file and output as a matrix or output the state. The characters in the file are 0 or 1 representing if a cell is empty.
\item
The read() function reads the input file and turn the characters to integers and
output as a 2 dimentional sequence.
\item
The write function takes a 2-dimentional sequence as input and write it into the file.
When finish writing one row, it goes to next line.
\end{itemize}
\subsection*{Local Functions}

\noindent split: $string \times char \rightarrow \mbox{(seq of string)}$\\
\noindent split ($str, delimiter$) $= internal$ \\



\newpage

\section* {Game Board Display Module}

\subsection*{Module}

Display

\subsection* {Uses}

N/A

\subsection* {Syntax}

\subsubsection* {Exported Constants}

None

\subsubsection* {Exported Types}

None

\subsubsection* {Exported Access Programs}

\begin{tabular}{| l | l | l | l |}
\hline
\textbf{Routine name} & \textbf{In} & \textbf{Out} & \textbf{Exceptions}\\
\hline
display & seq of (seq of $\mathbb{Z}$) & ~ & ~\\
\hline
\end{tabular}

\subsection* {Semantics}

\subsubsection* {State Variables}

None

\subsubsection* {State Invariant}

None

\subsubsection* {Access Routine Semantics}

display($total$):
\begin{itemize}
\item output: None
\item exception: None
\end{itemize}

\subsubsection* {Assumptions \& Design Decisions}
\begin{itemize}
\item
This module can display the state of game board on the console. The state of game
board is represented as 2-dimentional sequence of integers. And according to this
sequence, the display function can show each cell by printing the ASCII symbols.
\end{itemize}
\newpage




\section* {Game Board ADT Module}

\subsection*{Template Module}

Board

\subsection* {Uses}

None

\subsection* {Syntax}

\subsubsection* {Exported Access Programs}

\begin{tabular}{| l | l | l | l |}
\hline
\textbf{Routine name} & \textbf{In} & \textbf{Out} & \textbf{Exceptions}\\

\hline
Board  & seq of (seq of $\mathbb{Z}$)& GameBoardT & ~\\
\hline
get & $\mathbb{Z}$, $\mathbb{Z}$ &$\mathbb{Z}$ & ~\\

\hline
next & ~ & & ~\\
\hline
toSeq & ~ & seq of (seq of $\mathbb{Z}$) & \\

\hline
\end{tabular}

\subsection* {Semantics}

\subsubsection* {State Variables}

None

\subsubsection* {State Invariant}

None
\subsubsection* {Assumptions \& Design Decisions}

\begin{itemize}
\item The Board constructor is called before any other
access routine is called on that instance. Once a Board has been created, the constructor will not be called on it again.

\item get() is called when checking one of the cell. It takes two integers as the coordinate of a cell and output the integer representing its population.

\item The next() function is used to move the game board to the next stage. It checks the neighbours for every cell and determines if the cell should be populated in the next state. The rule is according to the population of neighbours.

\item The 2-dimentional sequence is used to represent for the game board. toSeq() can be used to get the sequence of the board, used to show on the console or write into file.
\end{itemize}

\subsubsection* {Access Routine Semantics}



\noindent Board($total$):
\begin{itemize}
\item transition: $total := total$
\item exception: none
\end{itemize}

\noindent get($row, col$):
\begin{itemize}
\item output:
$out := total[row][col]$
\item exception: none
\end{itemize}

\noindent next():
\begin{itemize}
\item output: none
\item transition:
\begin{tabular}{|p{2.5cm}|p{3cm}|l|}
\hhline{|-|-|-|}
total[$i$][$j$] = 1 & $count(i, j) < 2$ & total[$i$][$j$] := 0 \\
\hhline{|~|-|-|}
& $count(i, j) >3$ & total[$i$][$j$] := 0  \\
\hhline{|~|-|-|}
 & $count(i, j) = 2$& total[$i$][$j$] := 1  \\
 \hhline{|~|-|-|}
  & $count(i, j) = 3$& total[$i$][$j$] := 1  \\
\hhline{|-|-|-|}
total[$i$][$j$] = 0 & $count(i, j) > 3$ & total[$i$][$j$] := 0 \\
\hhline{|~|-|-|}
& $count(i, j) < 3$ & total[$i$][$j$] := 0  \\
\hhline{|~|-|-|}
 & $count(i, j) = 3$& total[$i$][$j$] := 1  \\
\hhline{|-|-|-|}
\end{tabular}
\item exception: none

\end{itemize}


\noindent toSeq():
\begin{itemize}
\item output: $out:= total$

\item exception: none

\end{itemize}



\subsection*{Local Functions}

\noindent count: $ \mathbb{Z} \times \mathbb{Z} \rightarrow \mathbb{Z}$\\
\noindent count ($row, col$) $= sum$ such that ($sum = total[row-1][col-1] + total[row-1][col] + total[row-1][col+1] + total[row][col-1] + total[row][col+1] + total[row+1][col-1] + total[row+1][col] + total[row+1][col+1])$\\


\end{document}
